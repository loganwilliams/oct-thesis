% $Log: abstract.tex,v $
% Revision 1.1  93/05/14  14:56:25  starflt
% Initial revision
% 
% Revision 1.1  90/05/04  10:41:01  lwvanels
% Initial revision
% 
%
%% The text of your abstract and nothing else (other than comments) goes here.
%% It will be single-spaced and the rest of the text that is supposed to go on
%% the abstract page will be generated by the abstractpage environment.  This
%% file should be \input (not \include 'd) from cover.tex.

In this thesis, the design and implementation of a fiber optic Doppler optical coherence microscopy (FO-DOCM) system for cochlear imaging applications is presented. The use of a fiber optic design significantly reduces system size and complexity and the construction of a novel alignment and micropositioning apparatus increases ease of use for the researcher performing the imaging. To enable precise motion measurements, a time domain DOCM approach is used, utilizing an acousto-optic modulator (AOM) based optical heterodyne system to generate a stationary interference carrier frequency. By referencing this interference signal against the AOM drive signals, precise measurements of motions with magnitude on the order of $10$ pm are shown to be possible. In addition to interferometrically measuring small amplitude motion, the FO-DOCM system is shown to be capable of imaging with a volumetric resolution of $10 \times 9 \times 9$ \micron. Demonstrative results of imaging cochlear tissue are presented by using the FO-DOCM to image and measure motion in a guinea pig cochlea {\em in vitro}.

% In this thesis, I designed and implemented a compiler which performs
% optimizations that reduce the number of low-level floating point operations
% necessary for a specific task; this involves the optimization of chains of
% floating point operations as well as the implementation of a ``fixed'' point
% data type that allows some floating point operations to simulated with integer
% arithmetic.  The source language of the compiler is a subset of C, and the
% destination language is assembly language for a micro-floating point CPU.  An
% instruction-level simulator of the CPU was written to allow testing of the
% code.  A series of test pieces of codes was compiled, both with and without
% optimization, to determine how effective these optimizations were.
