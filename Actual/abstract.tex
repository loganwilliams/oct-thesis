% $Log: abstract.tex,v $
% Revision 1.1  93/05/14  14:56:25  starflt
% Initial revision
% 
% Revision 1.1  90/05/04  10:41:01  lwvanels
% Initial revision
% 
%
%% The text of your abstract and nothing else (other than comments) goes here.
%% It will be single-spaced and the rest of the text that is supposed to go on
%% the abstract page will be generated by the abstractpage environment.  This
%% file should be \input (not \include 'd) from cover.tex.

In this thesis, I designed and implemented a fiber optic doppler optical coherence microscopy system for use in imaging the mammilian cochlea. The use of a fiber optic design significantly reduces system size and complexity and increases ease of use for the researcher performing the imaging. A time domain OCT approach is used. Doppler OCT is performed by using acousto-optic modulators to generate a carrier frequency, from which the amplitude and phase of any periodic motion of the sample can be found. The design of the system is discussed, and characterizations are performed to measure against theoretical design specifications.

% In this thesis, I designed and implemented a compiler which performs
% optimizations that reduce the number of low-level floating point operations
% necessary for a specific task; this involves the optimization of chains of
% floating point operations as well as the implementation of a ``fixed'' point
% data type that allows some floating point operations to simulated with integer
% arithmetic.  The source language of the compiler is a subset of C, and the
% destination language is assembly language for a micro-floating point CPU.  An
% instruction-level simulator of the CPU was written to allow testing of the
% code.  A series of test pieces of codes was compiled, both with and without
% optimization, to determine how effective these optimizations were.
